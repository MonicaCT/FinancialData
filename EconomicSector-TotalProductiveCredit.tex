\documentclass{beamer}
\usepackage{beamerthemesplit}
\usepackage{amsmath}
\usepackage[utf8]{inputenc}
\usepackage[T1]{fontenc}
\usepackage{graphicx}
\usepackage{multicol}
\makeatother    
\graphicspath{ {/Dropbox/InternationalMacro/} }
\title[Credit Type]{TOTAL PRODUCTIVE CREDIT}
%\author{Second version}
%\institute{}
%\subtitle{Universidad Mayor de San Andrés}
\date{\today}

\begin{document}
	\begin{frame}
		\begin{center}
			\titlepage{IDB Invest}
		\end{center}
	\end{frame}

%************************************************************************
%************************************************************************
%************************************************************************
\begin{frame}
    COUNTRIES
    \begin{multicols}{3}
	\begin{itemize}
		\item Bolivia
		\item Brazil
		\item Costa Rica
		\item Dominican Republic
		\item El Salvador
		\item Guatemala
		\item Honduras
		\item Mexico
		\item Nicaragua
		\item Panama
		\item Paraguay
		\item Peru
	\end{itemize}
\end{multicols}	
\end{frame}
%
%\begin{frame}
%	\begin{figure}
%		\centering
%		ARGENTINA: Total Productive Credit\\~\\
%		\footnote{Total credit is the sum of \textit{crédito empresarial, pyme, microcredito, hipotecario}, and \textit{crédito de consumo}.}
%		\includegraphics[width=1.2\textheight,height=1.2\textheight,keepaspectratio]{../Outputs.ES/Figures.ES/F-country/gipc.Argentina18}
%	\end{figure}
%\end{frame}
%
\begin{frame}
	\begin{figure}
		\centering
		BOLIVIA: Total Productive Credit\\~\\
		\includegraphics[width=1.2\textheight,height=1.2\textheight,keepaspectratio]{../Outputs.ES/Figures.ES/F-country/gipc.Bolivia18}
	\end{figure}
\end{frame}
%
\begin{frame}
	\begin{figure}
		\centering
		BRAZIL: Total Productive Credit\\~\\
		\includegraphics[width=1.2\textheight,height=1.2\textheight,keepaspectratio]{../Outputs.ES/Figures.ES/F-country/gipc.Brazil18}
	\end{figure}
\end{frame}
%
%\begin{frame}
%	\begin{figure}
%		\centering
%		CHILE: Total Productive Credit\\~\\
%		\includegraphics[width=1.2\textheight,height=1.2\textheight,keepaspectratio]{../Outputs.ES/Figures.ES/F-country/gipc.Chile18}
%	\end{figure}
%\end{frame}
%
\begin{frame}
	\begin{figure}
		\centering
		COSTA RICA: Total Productive Credit\\~\\
		\includegraphics[width=1.2\textheight,height=1.2\textheight,keepaspectratio]{../Outputs.ES/Figures.ES/F-country/gipc.CostaRica18}
	\end{figure}
\end{frame}
%
\begin{frame}
	\begin{figure}
		\centering
		DOMINICAN REPUBLIC: Total Productive Credit\\~\\
		\includegraphics[width=1.2\textheight,height=1.2\textheight,keepaspectratio]{../Outputs.ES/Figures.ES/F-country/gipc.DominicanRepublic18}
	\end{figure}
\end{frame}

\begin{frame}
\begin{figure}
		\centering
		EL SALVADOR: Total Productive Credit\\~\\
		\includegraphics[width=1.2\textheight,height=1.2\textheight,keepaspectratio]{../Outputs.ES/Figures.ES/F-country/gipc.ElSalvador18}
	\end{figure}
\end{frame}

\begin{frame}
	
	\begin{figure}
		\centering
		GUATEMALA: Total Productive Credit\\~\\
		\footnote{T.productive no incluye crédito en dólares}
		\includegraphics[width=1.2\textheight,height=1.2\textheight,keepaspectratio]{../Outputs.ES/Figures.ES/F-country/gipc.Guatemala18}
	\end{figure}
\end{frame}
%
\begin{frame}
	
	\begin{figure}
		\centering
		HONDURAS: Total Productive Credit\\~\\
		\footnote{En Honduras la diferencia entre el total productivo y el total por tipo de empresa es que los datos por tipo de empresa aun son preliminares}
		\includegraphics[width=1.2\textheight,height=1.2\textheight,keepaspectratio]{../Outputs.ES/Figures.ES/F-country/gipc.Honduras18}
	\end{figure}
\end{frame}
%
\begin{frame}
	
	\begin{figure}
		\centering
		MEXICO: Total Productive Credit\\~\\
		\includegraphics[width=1.2\textheight,height=1.2\textheight,keepaspectratio]{../Outputs.ES/Figures.ES/F-country/gipc.Mexico18}
	\end{figure}
\end{frame}
%
\begin{frame}
	
	\begin{figure}
		\centering
		NICARAGUA: Total Productive Credit\\~\\
		\includegraphics[width=1.2\textheight,height=1.2\textheight,keepaspectratio]{../Outputs.ES/Figures.ES/F-country/gipc.Nicaragua18}
\end{figure}
\end{frame}
%
\begin{frame}
	
	\begin{figure}
		\centering
		PANAMA: Total Productive Credit\\~\\
		\footnote{En T.productive incluye el crédito de consumo personal.}
		\includegraphics[width=1.2\textheight,height=1.2\textheight,keepaspectratio]{../Outputs.ES/Figures.ES/F-country/gipc.Panama18}
	\end{figure}
\end{frame}
%
\begin{frame}
	
	\begin{figure}
		\centering
		PARAGUAY: Total Productive Credit\\~\\
		\footnote{El total de crédito por actividad económica es el crédito bruto de todo el sistema financiero. El total por tipo de empresas es el crédito vigente del sistema bancario}
		\includegraphics[width=1.2\textheight,height=1.2\textheight,keepaspectratio]{../Outputs.ES/Figures.ES/F-country/gipc.Paraguay18}
	\end{figure}
\end{frame}
%
\begin{frame}
	
	\begin{figure}
		\centering
		PERU: Total Productive Credit\\~\\
		\includegraphics[width=1.2\textheight,height=1.2\textheight,keepaspectratio]{../Outputs.ES/Figures.ES/F-country/gipc.Peru18}
	\end{figure}
\end{frame}
%
%\begin{frame}
	
%	\begin{figure}
%		\centering
%		VENEZUELA: Total Productive Credit\\~\\
%		\includegraphics[width=1.2\textheight,height=1.2\textheight,keepaspectratio]{gi.Venezuela18}
%	\end{figure}
%\end{frame}
%

\begin{frame}
	
	THANK YOU!
	
\end{frame}

\end{document}
